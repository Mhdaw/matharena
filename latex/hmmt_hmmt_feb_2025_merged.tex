\subsection{1}
Compute the sum of the positive divisors (including $1$) of $9!$ that have units digit $1$.

\subsection{2}
Mark writes the expression $\sqrt{\underline{a b c d}}$ on the board, where $\underline{a b c d}$ is a four-digit number and $a \neq 0$. Derek, a toddler, decides to move the $a$, changing Mark's expression to $a \sqrt{\underline{b c d}}$. Surprisingly, these two expressions are equal. Compute the only possible four-digit number $\underline{a b c d}$.

\subsection{3}
Given that $x, y$, and $z$ are positive real numbers such that

$$
x^{\log _{2}(y z)}=2^{8} \cdot 3^{4}, \quad y^{\log _{2}(z x)}=2^{9} \cdot 3^{6}, \quad \text { and } \quad z^{\log _{2}(x y)}=2^{5} \cdot 3^{10}
$$
compute the smallest possible value of $x y z$.

\subsection{4}
Let $\lfloor z\rfloor$ denote the greatest integer less than or equal to $z$. Compute

$$
\sum_{j=-1000}^{1000}\left\lfloor\frac{2025}{j+0.5}\right\rfloor
$$

\subsection{5}
Let $\mathcal{S}$ be the set of all nonconstant monic polynomials $P$ with integer coefficients satisfying $P(\sqrt{3}+\sqrt{2})=$ $P(\sqrt{3}-\sqrt{2})$. If $Q$ is an element of $\mathcal{S}$ with minimal degree, compute the only possible value of $Q(10)-Q(0)$.

\subsection{6}
Let $r$ be the remainder when $2017^{2025!}-1$ is divided by 2025!. Compute $\frac{r}{2025!}$. (Note that $2017$ is prime.)

\subsection{7}
There exists a unique triple $(a, b, c)$ of positive real numbers that satisfies the equations
$$
2\left(a^{2}+1\right)=3\left(b^{2}+1\right)=4\left(c^{2}+1\right) \quad \text { and } \quad a b+b c+c a=1
$$

Compute $a+b+c$.

\subsection{8}
Define $\operatorname{sgn}(x)$ to be $1$ when $x$ is positive, $-1$ when $x$ is negative, and $0$ when $x$ is $0$. Compute

$$
\sum_{n=1}^{\infty} \frac{\operatorname{sgn}\left(\sin \left(2^{n}\right)\right)}{2^{n}}
$$
(The arguments to sin are in radians.)

\subsection{9}
Let $f$ be the unique polynomial of degree at most $2026$ such that for all $n \in\{1,2,3, \ldots, 2027\}$,
$$
f(n)= \begin{cases}1 & \text { if } n \text { is a perfect square } \\ 0 & \text { otherwise }\end{cases}
$$

Suppose that $\frac{a}{b}$ is the coefficient of $x^{2025}$ in $f$, where $a$ and $b$ are integers such that $\operatorname{gcd}(a, b)=1$. Compute the unique integer $r$ between $0$ and $2026$ (inclusive) such that $a-r b$ is divisible by $2027$. (Note that $2027$ is prime.)

\subsection{10}
Let $a, b$, and $c$ be pairwise distinct complex numbers such that
$$
a^{2}=b+6, \quad b^{2}=c+6, \quad \text { and } \quad c^{2}=a+6
$$

Compute the two possible values of $a+b+c$.

\subsection{11}
Compute the number of ways to arrange the numbers $1,2,3,4,5,6$, and $7$ around a circle such that the product of every pair of adjacent numbers on the circle is at most 20. (Rotations and reflections count as different arrangements.)

\subsection{12}
Kevin the frog in on the bottom-left lily pad of a $3 \times 3$ grid of lily pads, and his home is at the topright lily pad. He can only jump between two lily pads which are horizontally or vertically adjacent. Compute the number of ways to remove $4$ of the lily pads so that the bottom-left and top-right lily pads both remain, but Kelvin cannot get home.

\subsection{13}
Ben has $16$ balls labeled $1,2,3, \ldots, 16$, as well as $4$ indistinguishable boxes. Two balls are \emph{neighbors} if their labels differ by $1$. Compute the number of ways for him to put $4$ balls in each box such that each ball is in the same box as at least one of its neighbors. (The order in which the balls are placed does not matter.)

\subsection{14}
Sophie is at $(0,0)$ on a coordinate grid and would like to get to $(3,3)$. If Sophie is at $(x, y)$, in a single step she can move to one of $(x+1, y),(x, y+1),(x-1, y+1)$, or $(x+1, y-1)$. She cannot revisit any points along her path, and neither her $x$-coordinate nor her $y$-coordinate can ever be less than $0$ or greater than 3. Compute the number of ways for Sophie to reach $(3,3)$.

\subsection{15}
In an $11 \times 11$ grid of cells, each pair of edge-adjacent cells is connected by a door. Karthik wants to walk a path in this grid. He can start in any cell, but he must end in the same cell he started in, and he cannot go through any door more than once (not even in opposite directions). Compute the maximum number of doors he can go through in such a path.

\subsection{16}
Compute the number of ways to pick two rectangles in a $5 \times 5$ grid of squares such that the edges of the rectangles lie on the lines of the grid and the rectangles do not overlap at their interiors, edges, or vertices. The order in which the rectangles are chosen does not matter.

\subsection{17}
Compute the number of ways to arrange $3$ copies of each of the $26$ lowercase letters of the English alphabet such that for any two distinct letters $x_{1}$ and $x_{2}$, the number of $x_{2}$ 's between the first and second occurrences of $x_{1}$ equals the number of $x_{2}$ 's between the second and third occurrences of $x_{1}$.

\subsection{18}
Albert writes $2025$ numbers $a_{1}, \ldots, a_{2025}$ in a circle on a blackboard. Initially, each of the numbers is uniformly and independently sampled at random from the interval $[0,1]$. Then, each second, he \emph{simultaneously} replaces $a_{i}$ with $\max \left(a_{i-1}, a_{i}, a_{i+1}\right)$ for all $i=1,2, \ldots, 2025$ (where $a_{0}=a_{2025}$ and $a_{2026}=a_{1}$ ). Compute the expected value of the number of distinct values remaining after $100$ seconds.

\subsection{19}
Two points are selected independently and uniformly at random inside a regular hexagon. Compute the probability that a line passing through both of the points intersects a pair of opposite edges of the hexagon.

\subsection{20}
The circumference of a circle is divided into $45$ arcs, each of length $1$. Initially, there are $15$ snakes, each of length $1$, occupying every third arc. Every second, each snake independently moves either one arc left or one arc right, each with probability $\frac{1}{2}$. If two snakes ever touch, they merge to form a single snake occupying the arcs of both of the previous snakes, and the merged snake moves as one snake. Compute the expected number of seconds until there is only one snake left.

\subsection{21}
Equilateral triangles $\triangle A B C$ and $\triangle D E F$ are drawn such that points $B, E, F$, and $C$ lie on a line in this order, and point $D$ lies inside triangle $\triangle A B C$. If $B E=14, E F=15$, and $F C=16$, compute $A D$.

\subsection{22}
In a two-dimensional cave with a parallel floor and ceiling, two stalactites of lengths $16$ and $36$ hang perpendicularly from the ceiling, while two stalagmites of heights $25$ and $49$ grow perpendicularly from the ground. If the tips of these four structures form the vertices of a square in some order, compute the height of the cave.

\subsection{23}
Point $P$ lies inside square $A B C D$ such that the areas of $\triangle P A B, \triangle P B C, \triangle P C D$, and $\triangle P D A$ are 1, $2,3$, and $4$, in some order. Compute $P A \cdot P B \cdot P C \cdot P D$.

\subsection{24}
A semicircle is inscribed in another semicircle if the smaller semicircle's diameter is a chord of the larger semicircle, and the smaller semicircle's arc is tangent to the diameter of the larger semicircle.
Semicircle $S_{1}$ is inscribed in a semicircle $S_{2}$, which is inscribed in another semicircle $S_{3}$. The radii of $S_{1}$ and $S_{3}$ are $1$ and 10, respectively, and the diameters of $S_{1}$ and $S_{3}$ are parallel. The endpoints of the diameter of $S_{3}$ are $A$ and $B$, and $S_{2}$ 's arc is tangent to $A B$ at $C$. Compute $A C \cdot C B$.

\begin{tikzpicture}

    % S_1
    \coordinate (S_1_1) at (6.57,0.45);
    \coordinate (S_1_2) at (9.65,0.45);
    \draw (S_1_1) -- (S_1_2);

    \draw (S_1_1) arc[start angle=180, end angle=0, radius=1.54]
    node[midway,above] {};

    \node[above=0.5cm] at (7,1.2) {$S_1$};

    % S_2 
    \coordinate (S_2_1) at (6.32,4.82);
    \coordinate (S_2_2) at (9.95,0.68);

    \draw (S_2_1) -- (S_2_2);

    \draw (S_2_1) arc[start angle=131, end angle=311, radius=2.75]
    node[midway,above] {};

    \node[above=0.5cm] at (5,2) {$S_2$};

    % S_3
    \coordinate (A) at (0,0);
    \coordinate (B) at (10,0);
    \draw (A) -- (B);
    \fill (A) circle (2pt) node[below] {$A$};
    \fill (B) circle (2pt) node[below] {$B$};

    \draw (A) arc[start angle=180, end angle=0, radius=5]
    node[midway,above] {};
    \node[above=0.5cm] at (1,3) {$S_3$};

    \coordinate (C) at (8.3,0);
    \fill (C) circle (2pt) node[below] {$C$};

\end{tikzpicture}

\subsection{25}
Let $\triangle A B C$ be an equilateral triangle with side length $6$. Let $P$ be a point inside triangle $\triangle A B C$ such that $\angle B P C=120^{\circ}$. The circle with diameter $\overline{A P}$ meets the circumcircle of $\triangle A B C$ again at $X \neq A$. Given that $A X=5$, compute $X P$.

\subsection{26}
Trapezoid $A B C D$, with $A B \| C D$, has side lengths $A B=11, B C=8, C D=19$, and $D A=4$. Compute the area of the convex quadrilateral whose vertices are the circumcenters of $\triangle A B C, \triangle B C D$, $\triangle C D A$, and $\triangle D A B$.

\subsection{27}
Point $P$ is inside triangle $\triangle A B C$ such that $\angle A B P=\angle A C P$. Given that $A B=6, A C=8, B C=7$, and $\frac{B P}{P C}=\frac{1}{2}$, compute $\frac{[B P C]}{[A B C]}$.
(Here, $[X Y Z]$ denotes the area of $\triangle X Y Z$ ).

\subsection{28}
Let $A B C D$ be an isosceles trapezoid such that $C D>A B=4$. Let $E$ be a point on line $C D$ such that $D E=2$ and $D$ lies between $E$ and $C$. Let $M$ be the midpoint of $\overline{A E}$. Given that points $A, B, C, D$, and $M$ lie on a circle with radius $5$, compute $M D$.

\subsection{29}
Let $A B C D$ be a rectangle with $B C=24$. Point $X$ lies inside the rectangle such that $\angle A X B=90^{\circ}$. Given that triangles $\triangle A X D$ and $\triangle B X C$ are both acute and have circumradii $13$ and $15$, respectively, compute $A B$.

\subsection{30}
A plane $\mathcal{P}$ intersects a rectangular prism at a hexagon which has side lengths $45,66,63,55,54$, and 77, in that order. Compute the distance from the center of the rectangular prism to $\mathcal{P}$.

